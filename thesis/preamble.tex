
\documentclass[12pt,a4paper,twoside,openright]{book}
% ------------------------
%    Packages used
% ------------------------

%\usepackage{breqn} %not working
%mathtools % provide, for example dcases with nice vertical spaces
\usepackage{amsmath,amssymb, mathtools,mathrsfs,stmaryrd,titletoc}
\usepackage{latexsym}
\usepackage{appendix}
\usepackage[pdftex]{graphicx}
\usepackage{times}
%\usepackage{breqn}
\usepackage{dsfont}
\usepackage[usenames]{color}
\usepackage{authblk}
%\usepackage{epsfig}
%\usepackage{psfrag}
\let\Bbbk\relax
\usepackage[subscriptcorrection,slantedGreek,nofontinfo,mtphrd]{mtpro2}
\usepackage[retainorgcmds]{IEEEtrantools}
%\usepackage[dvips]{graphicx}
\usepackage{tabularx}
\usepackage{booktabs}
\usepackage{eucal}
\usepackage{setspace}
\usepackage[font=small,labelfont=md]{caption,subfig}
\usepackage[sort,nonamebreak, sectionbib]{natbib} 
\usepackage{multirow}
\usepackage[T1]{fontenc} % typing french
%\usepackage{fancyheadings}
\usepackage{fancyhdr}
\usepackage[bookmarks=true,colorlinks=true,linkcolor=red,citecolor=blue,backref=page]{hyperref}
\usepackage{makeidx}       % make index
\usepackage{float}         % make new float environment such as boxes (captioned)
\usepackage{listings}      % insert source code   
%\usepackage{matlab-prettifier} % insert Matlab
%\usepackage{bm}
%\usepackage{algorithmicx}
\usepackage{algorithm}
\usepackage{algpseudocode}

\usepackage{siunitx}


\usepackage[textsize=tiny]{todonotes}
\usepackage{nicefrac} % type inline fractions: \nicefrac{1}{2}
\usepackage{setspace}
\usepackage{lineno}  % write numbers for lines
%\usepackage[mediumspace,mediumqspace,Grey,squaren]{SIunits}
\usepackage{totcount} % to count the total number of references and other things
\usepackage[figure,table]{totalcount}
\usepackage{blkarray, bigstrut} % write complicated matrices with borders, see http://mirror.lagoon.nc/pub/ctan/macros/latex/contrib/blkarray/blkarray.pdf

\usepackage[norefs,nocites,ignoreunlbld]{refcheck} % warning for unreferred figs/tables/equas
% search in the .log file for unused fig to detect figures not referred to in the text.
\usepackage[activate={true,nocompatibility},final,tracking=true,kerning=true,spacing=true,factor=1100,stretch=10,shrink=10]{microtype}
\usepackage[capitalise]{cleveref} %Basically, cleveref must be loaded last.
% clever ref: instead of Fig.~\ref{d}, use \cref{d} or \Cref{d}  capitalise -> Figure 1
% \crefrange{eq1}{eq5}

%\usepackage[mediumspace,mediumqspace,Grey,squaren]{SIunits}
\usepackage{totcount} % to count the total number of references and other things
\usepackage[figure,table]{totalcount}
\usepackage{blkarray, bigstrut} % write complicated matrices with borders, see http://mirror.lagoon.nc/pub/ctan/macros/latex/contrib/blkarray/blkarray.pdf
\usepackage{setspace}
\usepackage{tikz}
\usetikzlibrary{arrows,decorations.pathmorphing,decorations.pathreplacing,backgrounds,positioning,fit,matrix,math,shapes.misc}
\tikzset{cross/.style={cross out, draw=black, minimum size=2*(#1-\pgflinewidth), inner sep=0pt, outer sep=0pt}, cross/.default={1pt}}

\usepackage{pgfplots}
 \pgfplotsset{compat=newest}
 %% the following commands are needed for some matlab2tikz features
 \usetikzlibrary{plotmarks}
 \usetikzlibrary{arrows.meta}
 \usepgfplotslibrary{patchplots}
 \usepackage{grffile}
\usepackage{gensymb} % for degree symbol
%\usepackage{underscore}
\usepackage[english]{babel}
\usepackage{imakeidx}
\usepackage{wasysym}

\sloppy

\makeatletter
\renewcommand\fs@ruled{\def\@fs@cfont{\bfseries}\let\@fs@capt\floatc@ruled
\def\@fs@pre{\hrule height 1.2pt depth0pt \kern2pt}%
%\def\@fs@post{\hrule height1.2pt depth0pt \kern2pt}%
\def\@fs@post{\kern2pt\hrule height 1.2pt depth0pt \kern2pt \relax}%
\def\@fs@mid{\kern2pt\hrule\kern2pt}%
\let\@fs@iftopcapt\iftrue}
\makeatother

% index generation
% see LATEX companion p 354
\newcommand{\bs}{\symbol{'134}}%print backslash
\newcommand{\Com}[1]{\texttt{\bs#1}%
\index{#1@\texttt{\bs#1}}}
\newcommand{\Prog}[1]{\texttt{#1}%
\index{#1@\texttt{#1} program }}

% shortcuts

% inner product <x,y>
\newcommand{\ipn}{\langle \cdot , \cdot \rangle}
\newcommand{\ip}[2]{\langle #1 , #2 \rangle}

% norm ||x||
\newcommand{\normn}{\left|\left| \cdot \right|\right|}
\newcommand{\norm}[1]{\left|\left|#1\right|\right|}
\newcommand{\meas}[1]{\left|#1\right|}

% McAuley brackets <x>
\newcommand{\mcauley}[1]{\langle #1 \rangle}

\newcommand{\x}{~$\times$~}
\newcommand{\fig}{Fig.~}
\newcommand{\eref}[1]{(\ref{eq:#1})}
\newcommand{\sref}[1]{\ref{section:#1}}
\newcommand{\fref}[1]{\ref{fig:#1}}
\newcommand{\tref}[1]{\ref{table:#1}}
%\newcommand{\eref}[1]{Eq.~(\ref{#1})}
%\newcommand{\erefs}[1]{Eqs.~(\ref{#1})}
%\newcommand{\fref}[1]{Fig.~\ref{#1}}
%\newcommand{\frefs}[1]{Figs.~\ref{#1}}
%\newcommand{\tref}[1]{Table~\ref{#1}}
\newcommand{\trefs}[1]{Tables~\ref{#1}}
%\newcommand{\sref}[1]{Section~\ref{#1}}
\newcommand{\srefs}[1]{Sections~\ref{#1}}
\newcommand{\crefs}[1]{Chapters~\ref{#1}}
\newcommand{\aref}[1]{Appendix~\ref{#1}}
\newcommand{\tsty}{\textstyle}
\newcommand{\dsty}{\displaystyle}
\newcommand{\D}{\displaystyle}
\newcommand{\arrow}{~$\rightarrow$~}
\newcommand{\otheta}{\overline \theta}
\newcommand{\mathG}{\mathcal{G}}

\newcommand{\mi}{_\mathrm{m}}
\newcommand{\ma}{_\mathrm{M}}

\newcommand{\Mi}{^\mathrm{m}}
\newcommand{\Ma}{^\mathrm{M}}

\newcommand{\dep}{_\mathrm{d}}
\newcommand{\ind}{_\mathrm{i}}

\newcommand{\di}{\mathrm{d}}

\newcommand{\defi}{\mathrel{\mathop:}=}

% abbreviations

\usepackage{xspace}
\newcommand{\eg}{\textit{e.g.}\xspace}
\newcommand{\ie}{\textit{i.e.},\xspace}
\newcommand{\etc}{\textit{etc.}\@\xspace}
\newcommand{\CF}{\textit{cf.}\,}     % i.e.
\newcommand{\cf}{\textit{cf.}\,}     % i.e.
\newcommand{\ETAL}{et. al.\@\xspace}
\newcommand{\etal}{et. al.\@\xspace}
\newcommand{\cmatrixb}{\left\{ \begin{matrix}}
\newcommand{\cmatrixe}{\end{matrix} \right\}}

% general vector/matrix commands:

\newcommand{\tvm}[1]{\textbf{#1}}
\newcommand{\tvms}[1]{$\boldsymbol{#1}$\ }
\newcommand{\vm}[1]{\mathbf{#1}}
\newcommand{\vms}[1]{\mathbf{#1}}
\newcommand{\bsym}[1]{\boldsymbol{#1}}

% vector/matrix for space coordinates 'x' and 'y'

\newcommand{\vx}{\mathbf{x}}
\newcommand{\vy}{\mathbf{y}}
\newcommand{\ve}[1]{\mathbf{e}_{#1}}
\newcommand{\bx}{\boldsymbol{x}}
\newcommand{\vxI}{\mathbf{x}_{I}}
\newcommand{\vj}[1]{\mathbf{#1}_{j}}
\newcommand{\xI}{x_{I}}
\newcommand{\yI}{y_{I}}
\newcommand{\hvx}{\hat{\mathbf{x}}}
\newcommand{\hx}{\hat{x}}
\newcommand{\hy}{\hat{y}}

\newcommand{\trans}{^\mathrm{T}}
\newcommand{\transi}{^\mathrm{-T}}
\newcommand{\el}{_\mathrm{e}}
\newcommand{\pl}{_\mathrm{p}}

\newcommand{\inte} [1]{\int_\Omega #1 d\Omega}
\newcommand{\intE}[1]{\int_{\Omega_0} #1 d\Omega_0}
\newcommand{\intg}[1]{\int_{\Gamma} #1 d\Gamma}
\newcommand{\intG}[1]{\int_{\Gamma_0} #1 d\Gamma_0}



%\newcommand{\T}{\underline{\vm{T}}}

% Shortcuts for making slides

\newcommand{\fontone}{\bfseries\Large}
\newcommand{\fonttwo}{\bfseries\large}
\newcommand{\fontonesc}{\scshape\Large}
\newcommand{\fonttwosc}{\scshape\large}
\newcommand{\fontthree}{\bfseries}
\newcommand{\bc}{\begin{center}}
\newcommand{\ec}{\end{center}}
\newcommand{\bitem}{\begin{itemize}}
\newcommand{\eitem}{\end{itemize}}

% ************************ MATH TYPE MACROS **************************
\newcommand{\mth}[1]{\mathit{#1}}      % print standard math italics
\newcommand{\boldsym}[1]{\mbox{\boldmath${#1}$}}
\newcommand{\vct}[1]{\boldsym{#1}}     % print vector
\newcommand{\fnc}[1]{\prno{#1}}        % print function i.e. sin ...
\newcommand{\mtx}[1]{\mathbf{#1}}      % print matrix
\newcommand{\msc}[1]{\mathcal{#1}}     % print script
\newcommand{\mss}[1]{\mathsf{#1}}      % print sans sarif
\newcommand{\tns}[1]{\boldsym{#1}}     % print tensor
\newcommand{\dbl}[1]{\mathbb{#1}}      % print sets letters like |R |C, etc...
%\newcommand{\Grad}{\stackrel{\rightharpoonup}{\boldsym{\nabla}}\!\!}
                                       % print gradient operator
\newcommand{\GradL}{\!\!\stackrel{\leftharpoonup}{\boldsym{\nabla}}}
                                       % print gradient operator
\newcommand{\Grad}{\boldsym{\nabla}}
                                       % print gradient operator
\newcommand{\Div}{\Grad\cdot}          % print divergnece operator
\newcommand{\DivL}{\cdot\GradL}        % print divergnece operator

\newcommand{\ljump}{\lbrack \! \lbrack } % print left jump oerator
\newcommand{\rjump}{\rbrack \! \rbrack } % print right jump oerator
                                         % derivative
\newcommand{\jump}[1]{\ljump {#1} \rjump} % jump operator
\newcommand{\derivv}[2]{ \frac{d^2 {#1} }{d {#2}$^2$ } }
\newcommand{\deriv}[2]{ \frac{d {#1} }{d {#2} } }   % partial derivivative
\newcommand{\pderiv}[2]{ \frac{\partial {#1} }{\partial {#2} } }
\newcommand{\testspc}{\mathcal{V}}
\newcommand{\trialspc}{\mathcal{S}}

\newcommand{\volint}[1]{\int_{\Omega}\,{#1}\,dV} % integrals
\newcommand{\bndint}[2]{\int_{#1}\,{#2}\,dS}

\newcommand{\beq}{\begin{equation}}
\newcommand{\eeq}{\end{equation}}
\newcommand{\beqa}{\begin{eqnarray}}
\newcommand{\eeqa}{\end{eqnarray}}

%--My own definitions--

\newcommand{\reals}{{\mathbb R}}
\newcommand{\dof}{\emph{dof}}
\newcommand{\dofs}{\emph{dofs}}
\newcommand{\bfalfi}{\mbox{\boldmath$\alpha$\unboldmath$_i$}}
\newcommand{\bfalfj}{\mbox{\boldmath$\alpha$\unboldmath$_j$}}
\newcommand{\remark}[2]{\vspace{0.1cm}
\noindent {\bf Remark #1:}  {#2}  \vspace{0.1cm}}
\newcommand{\invisible}[1]{}
\newcommand{\bgl}[1]{\mbox{\boldmath$#1$\unboldmath}}
\newcommand{\parti}[2]{\frac{\partial #1}{\partial #2}}

% X-FEM related
\newcommand{\xfemlong}{\textit{eXtended Finite Element Method}\,}     % extend..
\newcommand{\xfem}{\textit{X-FEM}\,}     % x-fem

% ********************** VERBATIM AND IGNORE *************************
\newcommand{\bv}{\begin{verbatim}}
\newcommand{\V}{\verb}                  % EX: \V=-d{#@~}= Expr must
                                        % fit on a line

% ************************ FIGURE COMMANDS ***************************
\newcommand{\testpix}[1]{\fbox{\begin{minipage}[c]{\textwidth}
                      #1 \end{minipage} }}

%\ifpdf
% \newcommand{\putfig}[2]{\includegraphics[scale=#2]{#1.pdf}}
%\else
% \newcommand{\putfig}[2]{\includegraphics[scale=#2]{#1.eps}}
%\fi

\newcommand{\putpstex}[1]{\includegraphics{#1.pstex_t}}
% END FIGURE COMMANDS


% cleverref package
%\crefname{figure}{Fig.}{Figs.}
%\crefname{equation}{Equation}{Equations}

\graphicspath{{./figures/}} 

\setlength{\oddsidemargin}{3mm}
\setlength{\evensidemargin}{-3mm}
\setlength{\textwidth}{160mm}
\setlength{\textheight}{220mm}

\newcommand\Algphase[1]{%
   \vspace*{-.7\baselineskip}\Statex\hspace*{\dimexpr-\algorithmicindent-2pt\relax}%\rule{\textwidth}{0.4pt}%
      \Statex\hspace*{-\algorithmicindent}\textbf{#1}%
      \vspace*{-.7\baselineskip}\Statex\hspace*{\dimexpr-\algorithmicindent-2pt\relax}%\rule{\textwidth}{0.4pt}%
}

% for blocks in Algorithm
\algblockdefx{Start}{End}[1]{\textbf{#1}}{\textbf{end}}

% -------------------------
%     Header and footer
% -------------------------

%\pagestyle{fancyplain}
%%\addtolength{\headwidth}{\marginparsep}
%%\addtolength{\headwidth}{\marginparwidth}
%\renewcommand{\chaptermark}[1]{\markboth{#1}{}}
%\renewcommand{\sectionmark}[1]{\markright{\ #1}}
%\lhead[\fancyplain{}{\bfseries\thepage}]{\fancyplain{}{\bfseries
%Sec. \thesection}}
%\chead[\fancyplain{}{\bfseries\leftmark}]{\fancyplain{}{\bfseries\rightmark}}
%\rhead[\fancyplain{}{\bfseries Chap.\
%\thechapter}]{\fancyplain{}{\bfseries\thepage}} \cfoot{}

\floatstyle{ruled}
\newfloat{Fbox}{thp}{lop}[section]
\floatname{Fbox}{Box}

%\theoremstyle{remark}                                                                                                       
\newtheorem{thm}{Theorem}[section]                                                                                          
\newtheorem{rmk}[thm]{Remark}    


\lstloadlanguages{C++,Matlab,Python}
\definecolor{mygreen}{rgb}{0,0.6,0}
\definecolor{darkgray}{rgb}{0.95,0.95,0.95}
\lstset{backgroundcolor=\color{darkgray},
  basicstyle=\color{red}\ttfamily,
  keywordstyle=\color{blue}\bfseries
}

\lstdefinelanguage{Sage}{%
    language     = Python,
   morekeywords={var, latex,  view, show, diff, integral, simplify, simplify_full}
}

\lstdefinelanguage{Julia}%
  {morekeywords={abstract,break,case,catch,const,continue,do,else,elseif,%
   end,export,false,for,function,immutable,import,importall,if,in,%
   macro,module,otherwise,quote,return,switch,true,try,type,typealias,%
   using,while,Float,Float64,tic,toc,Array,zeros, det, length, size, new},%
   sensitive=true,%
   alsoother={$},%
   morecomment=[l]\#,%
   morecomment=[n]{\#=}{=\#},%
   morestring=[s]{"}{"},%
   morestring=[m]{'}{'},%
   }[keywords,comments,strings]%

\lstdefinestyle{julia}
{
 basicstyle=\scriptsize, numbers=left, numberstyle=\tiny,%
 showstringspaces=false, language=Julia, escapechar=|,frame=tb,%
 commentstyle=\color{mygreen},
 morekeywords={cell, ones, repmat, intersect, unique, Float, FLoat64}
}

\lstdefinestyle{C++}
{
 basicstyle=\footnotesize, numbers=none, numberstyle=\tiny,%
 showstringspaces=false, language=C++, escapechar=|,frame=tb,
 commentstyle=\color{mygreen}
}

\lstdefinestyle{C-numbered}
{
 basicstyle=\footnotesize, numbers=left, numberstyle=\tiny,%
 showstringspaces=false, language=C++, escapechar=|,frame=tb,
 commentstyle=\color{mygreen}
}

\lstdefinestyle{matlab}
{
 basicstyle=\footnotesize, numbers=left, numberstyle=\tiny,%
 showstringspaces=false, language=Matlab, escapechar=|,frame=tb,%
 commentstyle=\color{mygreen},
 morekeywords={cell, ones, repmat, intersect, unique}
}

\lstdefinestyle{Sage}
{
 basicstyle=\footnotesize, numbers=left, numberstyle=\tiny,%
 showstringspaces=false, language=Sage, escapechar=|,frame=tb,%
 commentstyle=\color{mygreen}
}

%\lstset{language=Matlab,
%       basicstyle=\footnotesize,
%       keywordstyle=\color{red},
%       commentstyle=\itshape,
%       stringstyle=\ttfamily,
%       showstringspaces=false,
%       tabsize=2}
%\lstset{backgroundcolor=\color{MyDarkBlue}}
\lstdefinestyle{commentstyle}{color=\color{green}}

\renewcommand{\algorithmicrequire}{\textbf{Input:}}
\renewcommand{\algorithmicensure}{\textbf{Output:}}

\lstdefinestyle{tex}
{
 basicstyle=\footnotesize, numbers=left, numberstyle=\tiny,%
 showstringspaces=false, language=TeX, escapechar=|,frame=tb
}

%-----------------------------------------------------------------------
\newcommand{\tty}[1]{\textnormal{\texttt{#1}}}
\newcommand{\sym}[1]{\textnormal{\textit{#1}}}

\lstnewenvironment{snippet}[1][]
{
 \lstset{style=matlab, xleftmargin=5mm, gobble=4, #1}
}
{}

\lstnewenvironment{snippet1}[1][]
{
 \lstset{style=matlab, xleftmargin=5mm, gobble=4, #1}
}
{}

\lstnewenvironment{code-julia}[1][]
{
 \lstset{style=julia, xleftmargin=5mm, gobble=4, #1}
}
{}

\lstnewenvironment{code-sage}[1][]%
{
   \noindent
   \minipage{\linewidth} 
   \vspace{0.5\baselineskip}
  \lstset{style=Sage, xleftmargin=5mm, gobble=4, #1}
}
{\endminipage}

\lstnewenvironment{snippetC}[1][]
{
 \lstset{style=C++, xleftmargin=5mm, gobble=4, #1}
}
{}

\lstnewenvironment{code-matlab}[1][]%
{
   \noindent
   \minipage{\linewidth} 
   \vspace{0.5\baselineskip}
  \lstset{style=matlab, xleftmargin=5mm, gobble=4, #1}
}
{\endminipage}

\lstnewenvironment{code-c}[1][]%
{
   \noindent
   \minipage{\linewidth} 
   \vspace{0.5\baselineskip}
  \lstset{style=C++, xleftmargin=5mm, gobble=4, #1}
}
{\endminipage}

\lstnewenvironment{code-c-num}[1][]%
{
   \noindent
   \minipage{\linewidth} 
   \vspace{0.5\baselineskip}
  \lstset{style=C-numbered, xleftmargin=5mm, gobble=4, #1}
}
{\endminipage}


\newcolumntype{C}{>{\centering\arraybackslash}X}

%%%% these patches ensure that the backrefs point to the actual occurrences of the citations in the text, not just the page or section in which they appeared
%%%% https://tex.stackexchange.com/questions/54541/precise-back-reference-target-with-hyperref-and-backref
%%%% BEGIN BACKREF DIRECT PATCH, apply these AFTER loading hyperref package with appropriate backref option
% The following options are provided for the patch, currently with a poor interface!
% * If there are multiple cites on the same (page|section) (depending on backref mode),
%   should we show only the first one or should we show them all?
\newif\ifbackrefshowonlyfirst
\backrefshowonlyfirstfalse
%\backrefshowonlyfirsttrue
%%%% end of options
%
% hyperref is essential for this patch to make any sense, so it is not unreasonable to request it be loaded before applying the patch
\makeatletter
% 1. insert a phantomsection before every cite, so hyperref has something to target
%    * in case natbib is loaded. hyperref provides an appropriate hook so this should be safe, and we don't even need to check if natbib is loaded!
\let\BR@direct@old@hyper@natlinkstart\hyper@natlinkstart
\renewcommand*{\hyper@natlinkstart}{\phantomsection\BR@direct@old@hyper@natlinkstart}% note that the anchor will appear after any brackets at the start of the citation, but that's not really a big issue?
%    * if natbib isn't used, backref lets \@citex to \BR@citex during \AtBeginDocument
%      so just patch \BR@citex
\let\BR@direct@oldBR@citex\BR@citex
\renewcommand*{\BR@citex}{\phantomsection\BR@direct@oldBR@citex}%

% 2. if using page numbers, show the page number but still hyperlink to the phantomsection instead of just the page!
\long\def\hyper@page@BR@direct@ref#1#2#3{\hyperlink{#3}{#1}}

% check which package option the user loaded (pages (hyperpageref) or sections (hyperref)?)
\ifx\backrefxxx\hyper@page@backref
    % they wanted pages! make sure they get our re-definition
    \let\backrefxxx\hyper@page@BR@direct@ref
    \ifbackrefshowonlyfirst
        %\let\backrefxxxdupe\hyper@page@backref% test only the page number
        \newcommand*{\backrefxxxdupe}[3]{#1}% test only the page number
    \fi
\else
    \ifbackrefshowonlyfirst
        \newcommand*{\backrefxxxdupe}[3]{#2}% test only the section name
    \fi
\fi

% 3. now make sure that even if there is no numbered section, the hyperref's still work instead of going to the start of the document!
\RequirePackage{etoolbox}
\patchcmd{\Hy@backout}{Doc-Start}{\@currentHref}{}{\errmessage{I can't seem to patch backref}}
\makeatother
%%%% END BACKREF PATCHES

\flushbottom
