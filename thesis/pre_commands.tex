\makeatletter
\renewcommand\fs@ruled{\def\@fs@cfont{\bfseries}\let\@fs@capt\floatc@ruled
\def\@fs@pre{\hrule height 1.2pt depth0pt \kern2pt}%
%\def\@fs@post{\hrule height1.2pt depth0pt \kern2pt}%
\def\@fs@post{\kern2pt\hrule height 1.2pt depth0pt \kern2pt \relax}%
\def\@fs@mid{\kern2pt\hrule\kern2pt}%
\let\@fs@iftopcapt\iftrue}
\makeatother

% index generation
% see LATEX companion p 354
\newcommand{\bs}{\symbol{'134}}%print backslash
\newcommand{\Com}[1]{\texttt{\bs#1}%
\index{#1@\texttt{\bs#1}}}
\newcommand{\Prog}[1]{\texttt{#1}%
\index{#1@\texttt{#1} program }}

% shortcuts

% inner product <x,y>
\newcommand{\ipn}{\langle \cdot , \cdot \rangle}
\newcommand{\ip}[2]{\langle #1 , #2 \rangle}

% norm ||x||
\newcommand{\normn}{\left|\left| \cdot \right|\right|}
\newcommand{\norm}[1]{\left|\left|#1\right|\right|}
\newcommand{\meas}[1]{\left|#1\right|}

% McAuley brackets <x>
\newcommand{\mcauley}[1]{\langle #1 \rangle}

\newcommand{\x}{~$\times$~}
\newcommand{\fig}{Fig.~}
\newcommand{\eref}[1]{(\ref{eq:#1})}
\newcommand{\sref}[1]{\ref{section:#1}}
\newcommand{\fref}[1]{\ref{fig:#1}}
\newcommand{\tref}[1]{\ref{table:#1}}
%\newcommand{\eref}[1]{Eq.~(\ref{#1})}
%\newcommand{\erefs}[1]{Eqs.~(\ref{#1})}
%\newcommand{\fref}[1]{Fig.~\ref{#1}}
%\newcommand{\frefs}[1]{Figs.~\ref{#1}}
%\newcommand{\tref}[1]{Table~\ref{#1}}
\newcommand{\trefs}[1]{Tables~\ref{#1}}
%\newcommand{\sref}[1]{Section~\ref{#1}}
\newcommand{\srefs}[1]{Sections~\ref{#1}}
\newcommand{\crefs}[1]{Chapters~\ref{#1}}
\newcommand{\aref}[1]{Appendix~\ref{#1}}
\newcommand{\tsty}{\textstyle}
\newcommand{\dsty}{\displaystyle}
\newcommand{\D}{\displaystyle}
\newcommand{\arrow}{~$\rightarrow$~}
\newcommand{\otheta}{\overline \theta}
\newcommand{\mathG}{\mathcal{G}}

\newcommand{\mi}{_\mathrm{m}}
\newcommand{\ma}{_\mathrm{M}}

\newcommand{\Mi}{^\mathrm{m}}
\newcommand{\Ma}{^\mathrm{M}}

\newcommand{\dep}{_\mathrm{d}}
\newcommand{\ind}{_\mathrm{i}}

\newcommand{\di}{\mathrm{d}}

\newcommand{\defi}{\mathrel{\mathop:}=}

% abbreviations

\usepackage{xspace}
\newcommand{\eg}{\textit{e.g.}\xspace}
\newcommand{\ie}{\textit{i.e.},\xspace}
\newcommand{\etc}{\textit{etc.}\@\xspace}
\newcommand{\CF}{\textit{cf.}\,}     % i.e.
\newcommand{\cf}{\textit{cf.}\,}     % i.e.
\newcommand{\ETAL}{et. al.\@\xspace}
\newcommand{\etal}{et. al.\@\xspace}
\newcommand{\cmatrixb}{\left\{ \begin{matrix}}
\newcommand{\cmatrixe}{\end{matrix} \right\}}

% general vector/matrix commands:

\newcommand{\tvm}[1]{\textbf{#1}}
\newcommand{\tvms}[1]{$\boldsymbol{#1}$\ }
\newcommand{\vm}[1]{\mathbf{#1}}
\newcommand{\vms}[1]{\mathbf{#1}}
\newcommand{\bsym}[1]{\boldsymbol{#1}}

% vector/matrix for space coordinates 'x' and 'y'

\newcommand{\vx}{\mathbf{x}}
\newcommand{\vy}{\mathbf{y}}
\newcommand{\ve}[1]{\mathbf{e}_{#1}}
\newcommand{\bx}{\boldsymbol{x}}
\newcommand{\vxI}{\mathbf{x}_{I}}
\newcommand{\vj}[1]{\mathbf{#1}_{j}}
\newcommand{\xI}{x_{I}}
\newcommand{\yI}{y_{I}}
\newcommand{\hvx}{\hat{\mathbf{x}}}
\newcommand{\hx}{\hat{x}}
\newcommand{\hy}{\hat{y}}

\newcommand{\trans}{^\mathrm{T}}
\newcommand{\transi}{^\mathrm{-T}}
\newcommand{\el}{_\mathrm{e}}
\newcommand{\pl}{_\mathrm{p}}

\newcommand{\inte} [1]{\int_\Omega #1 d\Omega}
\newcommand{\intE}[1]{\int_{\Omega_0} #1 d\Omega_0}
\newcommand{\intg}[1]{\int_{\Gamma} #1 d\Gamma}
\newcommand{\intG}[1]{\int_{\Gamma_0} #1 d\Gamma_0}



%\newcommand{\T}{\underline{\vm{T}}}

% Shortcuts for making slides

\newcommand{\fontone}{\bfseries\Large}
\newcommand{\fonttwo}{\bfseries\large}
\newcommand{\fontonesc}{\scshape\Large}
\newcommand{\fonttwosc}{\scshape\large}
\newcommand{\fontthree}{\bfseries}
\newcommand{\bc}{\begin{center}}
\newcommand{\ec}{\end{center}}
\newcommand{\bitem}{\begin{itemize}}
\newcommand{\eitem}{\end{itemize}}

% ************************ MATH TYPE MACROS **************************
\newcommand{\mth}[1]{\mathit{#1}}      % print standard math italics
\newcommand{\boldsym}[1]{\mbox{\boldmath${#1}$}}
\newcommand{\vct}[1]{\boldsym{#1}}     % print vector
\newcommand{\fnc}[1]{\prno{#1}}        % print function i.e. sin ...
\newcommand{\mtx}[1]{\mathbf{#1}}      % print matrix
\newcommand{\msc}[1]{\mathcal{#1}}     % print script
\newcommand{\mss}[1]{\mathsf{#1}}      % print sans sarif
\newcommand{\tns}[1]{\boldsym{#1}}     % print tensor
\newcommand{\dbl}[1]{\mathbb{#1}}      % print sets letters like |R |C, etc...
%\newcommand{\Grad}{\stackrel{\rightharpoonup}{\boldsym{\nabla}}\!\!}
                                       % print gradient operator
\newcommand{\GradL}{\!\!\stackrel{\leftharpoonup}{\boldsym{\nabla}}}
                                       % print gradient operator
\newcommand{\Grad}{\boldsym{\nabla}}
                                       % print gradient operator
\newcommand{\Div}{\Grad\cdot}          % print divergnece operator
\newcommand{\DivL}{\cdot\GradL}        % print divergnece operator

\newcommand{\ljump}{\lbrack \! \lbrack } % print left jump oerator
\newcommand{\rjump}{\rbrack \! \rbrack } % print right jump oerator
                                         % derivative
\newcommand{\jump}[1]{\ljump {#1} \rjump} % jump operator
\newcommand{\derivv}[2]{ \frac{d^2 {#1} }{d {#2}$^2$ } }
\newcommand{\deriv}[2]{ \frac{d {#1} }{d {#2} } }   % partial derivivative
\newcommand{\pderiv}[2]{ \frac{\partial {#1} }{\partial {#2} } }
\newcommand{\testspc}{\mathcal{V}}
\newcommand{\trialspc}{\mathcal{S}}

\newcommand{\volint}[1]{\int_{\Omega}\,{#1}\,dV} % integrals
\newcommand{\bndint}[2]{\int_{#1}\,{#2}\,dS}

\newcommand{\beq}{\begin{equation}}
\newcommand{\eeq}{\end{equation}}
\newcommand{\beqa}{\begin{eqnarray}}
\newcommand{\eeqa}{\end{eqnarray}}

%--My own definitions--

\newcommand{\reals}{{\mathbb R}}
\newcommand{\dof}{\emph{dof}}
\newcommand{\dofs}{\emph{dofs}}
\newcommand{\bfalfi}{\mbox{\boldmath$\alpha$\unboldmath$_i$}}
\newcommand{\bfalfj}{\mbox{\boldmath$\alpha$\unboldmath$_j$}}
\newcommand{\remark}[2]{\vspace{0.1cm}
\noindent {\bf Remark #1:}  {#2}  \vspace{0.1cm}}
\newcommand{\invisible}[1]{}
\newcommand{\bgl}[1]{\mbox{\boldmath$#1$\unboldmath}}
\newcommand{\parti}[2]{\frac{\partial #1}{\partial #2}}

% X-FEM related
\newcommand{\xfemlong}{\textit{eXtended Finite Element Method}\,}     % extend..
\newcommand{\xfem}{\textit{X-FEM}\,}     % x-fem

% ********************** VERBATIM AND IGNORE *************************
\newcommand{\bv}{\begin{verbatim}}
\newcommand{\V}{\verb}                  % EX: \V=-d{#@~}= Expr must
                                        % fit on a line

% ************************ FIGURE COMMANDS ***************************
\newcommand{\testpix}[1]{\fbox{\begin{minipage}[c]{\textwidth}
                      #1 \end{minipage} }}

%\ifpdf
% \newcommand{\putfig}[2]{\includegraphics[scale=#2]{#1.pdf}}
%\else
% \newcommand{\putfig}[2]{\includegraphics[scale=#2]{#1.eps}}
%\fi

\newcommand{\putpstex}[1]{\includegraphics{#1.pstex_t}}
% END FIGURE COMMANDS
